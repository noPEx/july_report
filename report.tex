\documentclass{article}

\title{Interoperability of fingerprint devices}
\begin{document}
\section*{Introduction}
For our experiment we have been using fingerprint scanner devices and their respective software development kits from three manufacturers
Lumidigm,Futronic and Secugen.
They have the following configuration.

\begin{center}
	\begin{tabular}{ | c | c | c | c | }
	\hline
	Specification & Lumidigm & Secugen & Futronic \\
	\hline
	Image Resolution & 500 DPI & 500 DPI & 500 DPI \\
	\hline
	Image Size & 352x544 px & 260x300 px & 320x480 px \\
	\hline
	Image Gray Scale & 8-bit gray level & 8-bit gray level & 8-bit gray-level \\
	\hline	
	\end{tabular}
\end{center}

By interoprability we mean given a biometric recognition system from a particular manufacturer we must be able to use it with other manufacturers as well
as one system from a particular manufacturer is not feasible to deply everywhere.


We have various expections from a interoperable system. Most importantly a person making enrollment from a device X, when  presents his biometrics data at biometric Y we should be verify his identity with a decent accuracy. As every biometric system has various processes that begins from image acquisition to matching using biometric templates other expectations would be to use the artifacts obtained from a biometric identity should be usable for the next stage by another biometric system.

\newpage
\section*{Achieving Interoperability}
We noted that all the images we acquired from the sensors are of the same format. The only difference is the dimension of the image acquired from different sensor. So to achieve interoprability at the feature extractor level we need to make sure that the image needs to be of the same size according to the requirements of the feature extractor. We observed that not all SDK has this feature.
Once we have an image we store the features from it. The format in which we store the feature should be consistent across the SDKs. Again we make a comment  here that not all the SDK does so. In particular we need to store the features in ISO/IEC 19794-2:2005 standard. It also has the advantage that we can store some information about the source of the image. That will be helpful to us in further stages.
We  observed that while matching the accuracy of the respective matcher of the SDKs is fine when we use it to compare features that are obtained from two instances from the same device. This accuracy is lost when we try to do verification with features obtained from images of different devices. So we need a correctional factor when we need to do matching in which we are using artifacts from two different sources.
So overall we are trying to achieve interoperability by making the following modifications :
\begin{itemize}
	\item By abiding to standards when storing a biometric trait i.e image and biometric template.
	\item Designing a interoperable matcher having property :
		\begin{itemize}
			\item Good accuracy regardless of the source of the biometric template as long as it follows to standards.
			\item Have mechanism to identify the source of the image and using this information to add correctional factor.
			\item Training our system and getting correction factor for scores if needed.
		\end{itemize}
\end{itemize}


\end{document}
